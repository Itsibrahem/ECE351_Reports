\documentclass[12pt,a4paper]{article}
\usepackage[utf8]{inputenc}
\usepackage[greek,english]{babel}
\usepackage{alphabeta} 
\usepackage[pdftex]{graphicx}
\usepackage[top=1in, bottom=1in, left=1in, right=1in]{geometry}
\linespread{1.06}
\setlength{\parskip}{8pt plus2pt minus2pt}
\widowpenalty 10000
\clubpenalty 10000
\newcommand{\eat}[1]{}
\newcommand{\HRule}{\rule{\linewidth}{0.5mm}}
\usepackage[official]{eurosym}
\usepackage{enumitem}
\setlist{nolistsep,noitemsep}
\usepackage[hidelinks]{hyperref}
\usepackage{cite}
\usepackage{lipsum}

%===========================================================
\usepackage{graphicx}
\usepackage{listings}

\renewcommand\lstlistingname{Source Code}

\usepackage{xcolor}
\definecolor{codegreen}{rgb}{0,0.6,0}
\definecolor{codegray}{rgb}{0.5,0.5,0.5}
\definecolor{codeblue}{rgb}{0,0,0.95}
\definecolor{backcolour}{rgb}{0.95,0.95,0.92}

\lstdefinestyle{mystyle}{
   backgroundcolor=\color{backcolour},   
   commentstyle=\color{codegreen},
   keywordstyle=\color{codeblue},
   numberstyle=\tiny\color{codegray},
   stringstyle=\color{codegreen},
   basicstyle=\ttfamily\footnotesize,
   breakatwhitespace=false,         
   breaklines=true,                 
   captionpos=b,                    
   keepspaces=true,                 
   numbers=left,                    
   numbersep=5pt,                  
   showspaces=false,                
   showstringspaces=false,
   showtabs=false,                  
   tabsize=2
   xleftmargin=10pt
}
\lstset{style=mystyle}

%===========================================================
\begin{document}
%===========================================================
\begin{titlepage}
\begin{center}
% Top 
\includegraphics[width=0.55\textwidth]{cut-logo-en}~\\[2cm]
% Title
\HRule \\[0.4cm]
{ \LARGE 
  \textbf{Lab 11 Project Report for ECE 351}\\[0.4cm]
  \emph{Z - Transform Operations}\\[0.4cm]
}
\HRule \\[1.5cm]
% Author
{ \large
  Ibrahem Alobaid \\[0.1cm]
  November 17, 2022\\[0.1cm]
  %#\texttt{user@cut.ac.cy}
}
\vfill
%\textsc{\Large Cyprus University of Technology}\\[0.4cm]\textsc{\large Department of Electrical Engineering,\\Computer Engineering \& Informatics}\\[0.4cm]
% Bottom

\end{center}
\end{titlepage}
%\begin{abstract}
%\lipsum[1-2]
%\addtocontents{toc}{\protect\thispagestyle{empty}}
%\end{abstract}
\newpage
%===========================================================
\tableofcontents
\addtocontents{toc}{\protect\thispagestyle{empty}}
\newpage
\setcounter{page}{1}
%===========================================================
%===========================================================
\section{Introduction}\label{sec:intro}
    In this lab, our purpose was to analyze discrete system using built-in functions and Christopher Felton developed function.
%===========================================================

\section{Equations}\label{sec:lit-rev}

    In this lab we were asked to consider the following causal function,

    $$y[k] = 2x[k] - 40x[k-1] + 10y[k-1] - 16y[k-2],$$
    
    where $y[k]$ is the output and x[k] is the input.

%===========================================================
\section{Methodology and Results}\label{sec:meth}

\begin{itemize}

\item \textbf{Part 1}\\
\begin{enumerate}

    \item
    For the first task we were asked to find $H(z)$ of the given causal function (shown in the equation section). 
    
    $$Y(z)(1 - 10Z^{-1} + 16z^{-2}) = X(z)(2 - 40z^{-1})$$
    
    $$H(z) = \frac{2z(z-20)}{z^2 - 10z +16}$$
    
    $$\frac{H(z)}{z} = \frac{2(z-20)}{z^2 - 10z +16}$$
    
    $$H(z) = \frac{6z}{z-2} - \frac{4z}{z-8}$$\\
%===========================================================

    \item 
    For the second task we found $h[k]$ by partial fraction expansion.
    
    $$h[k] = [6(2)^k - 4(8)^k]u[k]$$
\clearpage
%===========================================================

    \item
    In the third task we verified our result of $h[k]$ by using the scipy.signal.residuez() function.

\begin{lstlisting}[language=Python, caption={Task 3, Part 1}, label={lst:code}, mathescape=true, breaklines=true]
# ==================== Part 1 - Task 3 ====================
Num = [2, -40]
Den = [1, -10, 16]

R, P, K = sig.residuez(Num, Den)

print("H(z): using scipy.signal.residuez() function")
print("Residue = ", R, "\nPoles = ", P, "\nK = ", K)
\end{lstlisting}

The printed result,

\begin{figure}[h]
    \centering
    \includegraphics[width=0.5\textwidth]{r1.jpg}
\end{figure}\textbf{}
%===========================================================

    \item
    For the fourth task we used the zplane() function to get the pole-zero plot for $H(z)$.

\begin{lstlisting}[language=Python, caption={Task 4, Part 1}, label={lst:code}, mathescape=true, breaklines=true]
# ==================== Part 1 - Task 4 ====================
NumZ = [2, -40]
DenZ = [1, -10, 16]

Z, P, K = zplane(NumZ, DenZ)

print("\nH(z): using zplane() function")
print("Zeros = ", Z, "\nPoles = ", P, "\nK = ", K)
\end{lstlisting}

Resulted plot and printed value,

\begin{figure}[h]
    \centering
    \includegraphics[width=0.6\textwidth]{fig1.png}
\end{figure}\textbf{}
\clearpage

\begin{figure}[h]
    \centering
    \includegraphics[width=0.4\textwidth]{r2.jpg}
\end{figure}\textbf{}
%===========================================================

    \item
    In the fifth task we used scipy.signal.freqz() to get the mamgnitude and phase response of $H(z)$.

\begin{lstlisting}[language=Python, caption={Task 5, Part 1}, label={lst:code}, mathescape=true, breaklines=true]
# ==================== Part 1 - Task 5 ====================
w, h = sig.freqz(NumZ, DenZ, whole = True)

magDB = 20*np.log10(np.abs(h))
phase = np.angle(h)*180/np.pi
f = w/(2*np.pi)
\end{lstlisting}

The resulted plot,

\begin{figure}[h]
    \centering
    \includegraphics[width=0.7\textwidth]{fig2.png}
\end{figure}\textbf{}
\end{enumerate}   
\end{itemize}

%===========================================================
\section{Questions}\label{sec:res}
\begin{enumerate}
    \item 
    Looking at the plot generated in Task 4, is H(z) stable? Explain why or why not.\\
    
    \\From the plot shown in (\textbf{Task 4}), we see that $H(z)$ is unstable as the poles are shown to the right of the circle.\\
    
    \item Leave any feedback on the clarity of lab tasks, expectations, and deliverables.\\
    
    \\The lab instructions were clear and precise.
\end{enumerate}

%===========================================================   
\section{Conclusion}\label{sec:res}
    
    By completing the lab we have became familiar with how to analyze discrete system using Christopher Felton function and Python built-in functions.
 
%\lipsum[7-8]\cite{knuthwebsite}
%===========================================================
%===========================================================
\bibliographystyle{ieeetr}
\bibliography{refs}
\end{document} 
