\documentclass[12pt,a4paper]{article}
\usepackage[utf8]{inputenc}
\usepackage[greek,english]{babel}
\usepackage{alphabeta} 
\usepackage[pdftex]{graphicx}
\usepackage[top=1in, bottom=1in, left=1in, right=1in]{geometry}
\linespread{1.06}
\setlength{\parskip}{8pt plus2pt minus2pt}
\widowpenalty 10000
\clubpenalty 10000
\newcommand{\eat}[1]{}
\newcommand{\HRule}{\rule{\linewidth}{0.5mm}}
\usepackage[official]{eurosym}
\usepackage{enumitem}
\setlist{nolistsep,noitemsep}
\usepackage[hidelinks]{hyperref}
\usepackage{cite}
\usepackage{lipsum}

%===========================================================
\usepackage{graphicx}
\usepackage{listings}

\renewcommand\lstlistingname{Source Code}

\usepackage{xcolor}
\definecolor{codegreen}{rgb}{0,0.6,0}
\definecolor{codegray}{rgb}{0.5,0.5,0.5}
\definecolor{codeblue}{rgb}{0,0,0.95}
\definecolor{backcolour}{rgb}{0.95,0.95,0.92}

\lstdefinestyle{mystyle}{
   backgroundcolor=\color{backcolour},   
   commentstyle=\color{codegreen},
   keywordstyle=\color{codeblue},
   numberstyle=\tiny\color{codegray},
   stringstyle=\color{codegreen},
   basicstyle=\ttfamily\footnotesize,
   breakatwhitespace=false,         
   breaklines=true,                 
   captionpos=b,                    
   keepspaces=true,                 
   numbers=left,                    
   numbersep=5pt,                  
   showspaces=false,                
   showstringspaces=false,
   showtabs=false,                  
   tabsize=2
   xleftmargin=10pt
}
\lstset{style=mystyle}

%===========================================================
\begin{document}
%===========================================================
\begin{titlepage}
\begin{center}
% Top 
\includegraphics[width=0.55\textwidth]{cut-logo-en}~\\[2cm]
% Title
\HRule \\[0.4cm]
{ \LARGE 
  \textbf{Lab 12 Project Report for ECE 351}\\[0.4cm]
  \emph{Filter Design}\\[0.4cm]
}
\HRule \\[1.5cm]
% Author
{ \large
  Ibrahem Alobaid \\[0.1cm]
  December 8, 2022\\[0.1cm]
  %#\texttt{user@cut.ac.cy}
}
\vfill
%\textsc{\Large Cyprus University of Technology}\\[0.4cm]\textsc{\large Department of Electrical Engineering,\\Computer Engineering \& Informatics}\\[0.4cm]
% Bottom

\end{center}
\end{titlepage}
%\begin{abstract}
%\lipsum[1-2]
%\addtocontents{toc}{\protect\thispagestyle{empty}}
%\end{abstract}
\newpage
%===========================================================
\tableofcontents
\addtocontents{toc}{\protect\thispagestyle{empty}}
\newpage
\setcounter{page}{1}
%===========================================================
%===========================================================
\section{Introduction}\label{sec:intro}

    In this lab, our purpose was to filter a noisy signal. A position sensor signal is given to a positioning control system of an aircraft. Due to noise on the sensor signal, the positioning system is not receiving accurate reading. 
    
    The position measurement information is indicated to be in the range of $1.8kHz \leq f \leq 2.0kHz$. We are tasked to first identify the following:
    
\begin{itemize}
    \item
    Magnitudes and corresponding frequencies of the position measurement information.\\
    
    \item 
    Noise magnitudes and corresponding frequencies due to low frequency vibrations.\\
    
    \item
    Noise magnitudes and corresponding frequencies due to switching amplifier.\\
\end{itemize}
    
    Our second task is to filter the noisy signal, passing only the position measurement information. Our filter circuit required the following specifications:
    
\begin{itemize}
    \item 
    The position measurement information is attenuated by less than -0.3dB\\
    
    \item
    The low-frequency vibration noise must be attenuated by at least -30dB\\

    \item
    The switching amplifier noise must be attenuated by at least -21dB\\

    \item
    All noise that exists at frequencies greater than 100kHz must be completely attenuated (magnitudes less than 0.05V can be considered completely attenuated for all practical purposes in this situation)
\end{itemize}
\clearpage
%===========================================================

\section{Equations}\label{sec:lit-rev}

\begin{figure}[h]
    \centering
    \includegraphics[width=0.7\textwidth]{f1.png}
    \\Figure 1: Band-pass filter ($R = 1.3K\Omega,\ L = 1H,\ C = 7nF$)
\end{figure}\textbf{}
    The transfer function of the RLC circuit is,
    
    $$H(s) = \frac{k\beta s}{s^2 + \beta s + w_0^2} = \frac{(\frac{R}{L})s}{s^2 + (\frac{R}{L})s + \frac{1}{LC}}$$
    
    To design the RLC circuit, the inductor value was assumed to be $L = 1H$, corner frequencies were set to be $f_{c1} = 1.8kHz,\ f_{c2} = 2.0kHz$, and the following equations were used:

\begin{equation}
    w_c = 2\pi f_c
\end{equation}

\begin{equation}
    w_0 = \sqrt{w_{c1} \cdot w_{c2}}
\end{equation}

\begin{equation}
    \beta = w_{c2} - w_{c1}
\end{equation}

\begin{equation}
    w_0^2 = \frac{1}{LC}
\end{equation}

\begin{equation}
    \beta = \frac{R}{L}
\end{equation}

\clearpage
%===========================================================
\section{Methodology and Results}\label{sec:meth}

\begin{itemize}
    \item 
    Noisy signal plots:

\begin{figure}[h]
    \centering
    \includegraphics[width=0.7\textwidth]{f2.png}
    \\Figure 2: Sensor signal
\end{figure}

\begin{figure}[h]
    \centering
    \includegraphics[width=0.7\textwidth]{f3.png}
    \\Figure 3
\end{figure}
\clearpage

\begin{figure}[h]
    \centering
    \includegraphics[width=0.7\textwidth]{f4.png}
    \\Figure 4: Position measurements information
\end{figure}

\begin{figure}[h]
    \centering
    \includegraphics[width=0.7\textwidth]{f5.png}
    \\Figure 5: Noise due to low frequency vibrations
\end{figure}
\clearpage

\begin{figure}[h]
    \centering
    \includegraphics[width=0.7\textwidth]{f6.png}
    \\Figure 6: Noise due to switching amplifier
\end{figure}
%===========================================================

    \item
    Filter characteristics plots:
    
\begin{figure}[h]
    \centering
    \includegraphics[width=0.7\textwidth]{f7.png}
    \\Figure 7
\end{figure}
\clearpage

\begin{figure}[h]
    \centering
    \includegraphics[width=0.7\textwidth]{f8.png}
    \\Figure 8: Frequencies of position measurements information
\end{figure}

\begin{figure}[h]
    \centering
    \includegraphics[width=0.7\textwidth]{f9.png}
    \\Figure 9: Frequencies of Low-frequency vibrations
\end{figure}
\clearpage

\begin{figure}[h]
    \centering
    \includegraphics[width=0.7\textwidth]{f10.png}
    \\Figure 10: Frequencies of switching amplifier
\end{figure}
%===========================================================

    \item
    Filtered signal plots:

\begin{figure}[h]
    \centering
    \includegraphics[width=0.7\textwidth]{f11.png}
    \\Figure 11: Filtered sensor signal
\end{figure}    
\clearpage

\begin{figure}[h]
    \centering
    \includegraphics[width=0.7\textwidth]{f12.png}
    \\Figure 12
\end{figure}

\begin{figure}[h]
    \centering
    \includegraphics[width=0.7\textwidth]{f13.png}
    \\Figure 13: Filtered position measurements information
\end{figure}  
\clearpage

\begin{figure}[h]
    \centering
    \includegraphics[width=0.7\textwidth]{f14.png}
    \\Figure 14: Filtered low-frequency noise 
\end{figure} 

\begin{figure}[h]
    \centering
    \includegraphics[width=0.7\textwidth]{f15.png}
    \\Figure 15: Filtered high-frequency noise
\end{figure}
\clearpage
\end{itemize}

%===========================================================
\section{Questions}\label{sec:res}

\begin{enumerate}
    \item  
    Earlier this semester, you were asked what you personally wanted to get out of taking this course. Do you feel like that personal goal was met? Why or why not?\\
    
    I have check my \textbf{Lab 0}, and for some reason my answer was to be setup for next lab, so yes, I do feel like I have met that goal. Other than that, I feel comfortable using LATEX and Python which I have wanted to learn about.\\
    
    \item
    Please fill out the course feedback survey, I will read every word and very much appreciate the feedback.\\ 
    
    Will do.\\
    
    \item
    Good luck in the rest of your education and career!\\
    
    Thanks! Good luck to you too.

\end{enumerate}

%===========================================================   
\section{Conclusion}\label{sec:res}
    By completing the lab we have used many of the skills we learned about Python during the semester to characterize a noisy signal, design and characterize an RLC circuit to meet a given specifications, and filter out the noisy signal.
 
%\lipsum[7-8]\cite{knuthwebsite}
%===========================================================
%===========================================================
\bibliographystyle{ieeetr}
\bibliography{refs}
\end{document} 