\documentclass[12pt,a4paper]{article}
\usepackage[utf8]{inputenc}
\usepackage[greek,english]{babel}
\usepackage{alphabeta} 
\usepackage[pdftex]{graphicx}
\usepackage[top=1in, bottom=1in, left=1in, right=1in]{geometry}
\linespread{1.06}
\setlength{\parskip}{8pt plus2pt minus2pt}
\widowpenalty 10000
\clubpenalty 10000
\newcommand{\eat}[1]{}
\newcommand{\HRule}{\rule{\linewidth}{0.5mm}}
\usepackage[official]{eurosym}
\usepackage{enumitem}
\setlist{nolistsep,noitemsep}
\usepackage[hidelinks]{hyperref}
\usepackage{cite}
\usepackage{lipsum}

%===========================================================
\usepackage{graphicx}
\usepackage{listings}

\renewcommand\lstlistingname{Source Code}

\usepackage{xcolor}
\definecolor{codegreen}{rgb}{0,0.6,0}
\definecolor{codegray}{rgb}{0.5,0.5,0.5}
\definecolor{codeblue}{rgb}{0,0,0.95}
\definecolor{backcolour}{rgb}{0.95,0.95,0.92}

\lstdefinestyle{mystyle}{
   backgroundcolor=\color{backcolour},   
   commentstyle=\color{codegreen},
   keywordstyle=\color{codeblue},
   numberstyle=\tiny\color{codegray},
   stringstyle=\color{codegreen},
   basicstyle=\ttfamily\footnotesize,
   breakatwhitespace=false,         
   breaklines=true,                 
   captionpos=b,                    
   keepspaces=true,                 
   numbers=left,                    
   numbersep=5pt,                  
   showspaces=false,                
   showstringspaces=false,
   showtabs=false,                  
   tabsize=2
   xleftmargin=10pt
}
\lstset{style=mystyle}

%===========================================================
\begin{document}
%===========================================================
\begin{titlepage}
\begin{center}
% Top 
\includegraphics[width=0.55\textwidth]{cut-logo-en}~\\[2cm]
% Title
\HRule \\[0.4cm]
{ \LARGE 
  \textbf{Lab 10 Project Report for ECE 351}\\[0.4cm]
  \emph{Frequency Response}\\[0.4cm]
}
\HRule \\[1.5cm]
% Author
{ \large
  Ibrahem Alobaid \\[0.1cm]
  November 10, 2022\\[0.1cm]
  %#\texttt{user@cut.ac.cy}
}
\vfill
%\textsc{\Large Cyprus University of Technology}\\[0.4cm]\textsc{\large Department of Electrical Engineering,\\Computer Engineering \& Informatics}\\[0.4cm]
% Bottom

\end{center}
\end{titlepage}
%\begin{abstract}
%\lipsum[1-2]
%\addtocontents{toc}{\protect\thispagestyle{empty}}
%\end{abstract}
\newpage
%===========================================================
\tableofcontents
\addtocontents{toc}{\protect\thispagestyle{empty}}
\newpage
\setcounter{page}{1}
%===========================================================
%===========================================================
\section{Introduction}\label{sec:intro}
    In this lab, our purpose was to become familiar with Bode plots and frequency response tools using Python.
%===========================================================

\section{Equations}\label{sec:lit-rev}

\begin{figure}[h]
    \centering
    \includegraphics[width=0.7\textwidth]{RLC.jpg}
\end{figure}\textbf{}

We were given the following transfer function,

$$H(s) = \frac{\frac{1}{RC}s}{s^2 + \frac{1}{RC}s + \frac{1}{LC}}$$\\

and asked to find an expression for the magnitude and phase:

    $$|H(j\omega)| = \frac{\frac{\omega}{RC}}{\sqrt{(\frac{1}{LC}-\omega^2)^2+(\frac{\omega}{RC})^2}}$$
    
    $$\angle H(j\omega) = 90 - \arctan(\frac{\frac{\omega}{RC}}{\frac{1}{LC}-\omega^2})$$\\

We were also asked to plot the following signal,

    $$x(t) = cos(2\pi \cdot 100t) + cos(2\pi \cdot 3024t) + sin(2\pi \cdot 50000t)$$

\clearpage
%===========================================================
\section{Methodology and Results}\label{sec:meth}

\begin{itemize}

\item \textbf{Part 1}\\
\begin{enumerate}

    \item
    For the first task we were asked to plot the magnitude (in dB) and phase (in degrees) of the given RLC transfer function from $(10^3 \frac{rad}{s} \leq \omega \leq 10^6 \frac{rad}{s})$. For plotting the, we used matplotib.pyplot.plot() function to plot the x-axis on a logarithmic scale.\\

\begin{lstlisting}[language=Python, caption={Part 1, Task1}, label={lst:code}, mathescape=true, breaklines=true]
# ==================== Part 1 - Task 1 ====================
steps = 1
W = np. arange (1e3, 1e6 + steps, steps ) 

R = 1000
L = 27e-3
C = 100e-9

phase = 90-np.arctan((W/(R*C))/(1/(L*C)-(W**2)))*(180/np.pi)
for i in range(len(phase)):
    if (phase[i] > 90):
        phase[i] = phase[i] - 180

mag = (W/(R*C))/(np.sqrt((1/(L*C)-(W**2))**2+(W/(R*C))**2))
magDB = 20*np.log10(mag)
\end{lstlisting}

The resulting plot,

\begin{figure}[h]
    \centering
    \includegraphics[width=0.7\textwidth]{fig1.png}
\end{figure}\textbf{}
\clearpage

%===========================================================
    \item
    Our second task was to use scipy.signal.bode() function to plot the magnitude and phase frequency of the RLC transfer function (this should confirm our result from \textbf{Task 1}).\\

\begin{lstlisting}[language=Python, caption={Task 2, Part 1}, label={lst:code}, mathescape=true, breaklines=true]
# ==================== Part 1 - Task 2 ====================
num = [0, 1/(R*C), 0]
den = [1, 1/(R*C), 1/(L*C)]

W1, mag, phase = sig.bode((num,den), W)
\end{lstlisting}

The resulting plot,

\begin{figure}[h]
    \centering
    \includegraphics[width=0.7\textwidth]{fig2.png}
\end{figure}\textbf{}
\clearpage

%===========================================================
    \item
    For the third task we were asked to plot the frequency response of the given transfer function in $Hz$ within a specified range of frequencies.\\

\begin{lstlisting}[language=Python, caption={Task 3, Part 1}, label={lst:code}, mathescape=true, breaklines=true]
# ==================== Part 1 - Task 3 ====================
f = W/(2*np.pi)

plt.figure(figsize = (10, 7))

sys = con.TransferFunction(num, den)
_ = con.bode(sys, W, dB = True, Hz = True, deg = True, Plot = True)
\end{lstlisting}

The resulting plot,

\begin{figure}[h]
    \centering
    \includegraphics[width=0.7\textwidth]{fig3.png}
\end{figure}\textbf{}
\clearpage
\end{enumerate}

%===========================================================
\item \textbf{Part 2}\\
\begin{enumerate}

    \item
    For the first task we were asked to plot the following signal, 
    $$x(t) = cos(2\pi \cdot 100t) + cos(2\pi \cdot 3024t) + sin(2\pi \cdot 50000t)$$
    from $(0 \leq t \leq 0.01)$, a frequency high enough to capture all three frequencies, and a step size of $\frac{1}{f_s}$.\\

\begin{lstlisting}[language=Python, caption={Task 1, Part 2}, label={lst:code}, mathescape=true, breaklines=true]
# ==================== Part 2 - Task 1 ====================
fs = (2*50000*np.pi)
steps = 1 / fs
t = np.arange(0, 0.01+steps, steps)

x = np.cos(2*np.pi*100*t) + np.cos(2*np.pi*3024*t) + np.sin(2*np.pi*50000*t)
\end{lstlisting}

\begin{figure}[h]
    \centering
    \includegraphics[width=0.7\textwidth]{fig4.png}
\end{figure}\textbf{}

%===========================================================
    \item
    In the second task we converted the given input signal into the z-domain, using scipy.signal.bilinear(), to pass it through the RLC circuit.\\
    
\begin{lstlisting}[language=Python, caption={Task 2, Part 2}, label={lst:code}, mathescape=true, breaklines=true]
# ==================== Part 2 - Task 2 ====================
numZ, denZ = sig.bilinear(num, den, fs)
\end{lstlisting}
\clearpage

%===========================================================
    \item
    For the third task we used scipy.signal.lfilter() to pass the input signal through the RLC filter.\\
 
\begin{lstlisting}[language=Python, caption={Task 3, Part 2}, label={lst:code}, mathescape=true, breaklines=true]
# ==================== Part 2 - Task 3 ====================
y = sig.lfilter(numZ, denZ, x)
\end{lstlisting}  

%===========================================================
    \item
    For the forth task we plotted the output signal over the same period as \textbf{Task 1}.
    
\begin{figure}[h]
    \centering
    \includegraphics[width=0.7\textwidth]{fig5.png}
\end{figure}\textbf{}
\end{enumerate}   
\end{itemize}

%===========================================================
\section{Questions}\label{sec:res}
\begin{enumerate}
    \item Explain how the filter and filtered output in Part 2 makes sense given the Bode plots from Part 1. Discuss how the filter modifies specific frequency bands, in Hz.\\

    \\Looking at the results from \textbf{Part 1}, we observe that the RLC filter best passes signals with $f \approx 3K Hz$ (as can be seen in the Bode plot \textbf{Part 1 - Task 3}); Thus, passing input signal $x(t) = cos(2\pi \cdot 100t) + cos(2\pi \cdot 3024t) + sin(2\pi \cdot 50000t)$, the filtered signal $y(t)$ will be $cos(2\pi \cdot 3024t)$.\\

    \item  Discuss the purpose and workings of scipy.signal.bilinear() and scipy.signal.lfilter().\\
    
    \\The purpose of scipy.signal.bilinear() is to convert a given signal $x(t)$ into its equivalent z-domain.\\
    
    The purpose of scipy.signal.lfilter() is to pass a given input signal through a filter (transfer function in the z-domain).\\
    
    \item  What happens if you use a different sampling frequency in scipy.signal.bilinear() than you used for the time-domain signal?\\
    
    If we were to have a different value of $f_s$ in the scipy.signal.bilinear() call than the one we used in \textbf{Part 2 - Task 1}, the filtered signal will differ as we have changed the characteristics of the filter.\\
    
    \item Leave any feedback on the clarity of lab tasks, expectations, and deliverables.\\
    \\The lab instructions were clear and precise.
\end{enumerate}

%===========================================================   
\section{Conclusion}\label{sec:res}
    
    By completing the lab we have became familiar with frequency response tools and Bode plots using Python.
 
%\lipsum[7-8]\cite{knuthwebsite}
%===========================================================
%===========================================================
\bibliographystyle{ieeetr}
\bibliography{refs}
\end{document} 