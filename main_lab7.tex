\documentclass[12pt,a4paper]{article}
\usepackage[utf8]{inputenc}
\usepackage[greek,english]{babel}
\usepackage{alphabeta} 
\usepackage[pdftex]{graphicx}
\usepackage[top=1in, bottom=1in, left=1in, right=1in]{geometry}
\linespread{1.06}
\setlength{\parskip}{8pt plus2pt minus2pt}
\widowpenalty 10000
\clubpenalty 10000
\newcommand{\eat}[1]{}
\newcommand{\HRule}{\rule{\linewidth}{0.5mm}}
\usepackage[official]{eurosym}
\usepackage{enumitem}
\setlist{nolistsep,noitemsep}
\usepackage[hidelinks]{hyperref}
\usepackage{cite}
\usepackage{lipsum}

%===========================================================
\usepackage{graphicx}
\usepackage{listings}

\renewcommand\lstlistingname{Source Code}

\usepackage{xcolor}
\definecolor{codegreen}{rgb}{0,0.6,0}
\definecolor{codegray}{rgb}{0.5,0.5,0.5}
\definecolor{codeblue}{rgb}{0,0,0.95}
\definecolor{backcolour}{rgb}{0.95,0.95,0.92}

\lstdefinestyle{mystyle}{
   backgroundcolor=\color{backcolour},   
   commentstyle=\color{codegreen},
   keywordstyle=\color{codeblue},
   numberstyle=\tiny\color{codegray},
   stringstyle=\color{codegreen},
   basicstyle=\ttfamily\footnotesize,
   breakatwhitespace=false,         
   breaklines=true,                 
   captionpos=b,                    
   keepspaces=true,                 
   numbers=left,                    
   numbersep=5pt,                  
   showspaces=false,                
   showstringspaces=false,
   showtabs=false,                  
   tabsize=2
   xleftmargin=10pt
}
\lstset{style=mystyle}

%===========================================================
\begin{document}
%===========================================================
\begin{titlepage}
\begin{center}
% Top 
\includegraphics[width=0.55\textwidth]{cut-logo-en}~\\[2cm]
% Title
\HRule \\[0.4cm]
{ \LARGE 
  \textbf{Lab 7 Project Report for ECE 351}\\[0.4cm]
  \emph{Block Diagrams and System Stability}\\[0.4cm]
}
\HRule \\[1.5cm]
% Author
{ \large
  Ibrahem Alobaid \\[0.1cm]
  October 13, 2022\\[0.1cm]
  %#\texttt{user@cut.ac.cy}
}
\vfill
%\textsc{\Large Cyprus University of Technology}\\[0.4cm]\textsc{\large Department of Electrical Engineering,\\Computer Engineering \& Informatics}\\[0.4cm]
% Bottom

\end{center}
\end{titlepage}
%\begin{abstract}
%\lipsum[1-2]
%\addtocontents{toc}{\protect\thispagestyle{empty}}
%\end{abstract}
\newpage
%===========================================================
\tableofcontents
\addtocontents{toc}{\protect\thispagestyle{empty}}
\newpage
\setcounter{page}{1}
%===========================================================
%===========================================================
\section{Introduction}\label{sec:intro}
    In this lab, our purpose was to familiarize ourselves with Laplace-domain block diagrams and to use the transfer function factored form to determine a system stability. 
%===========================================================

\section{Equations}\label{sec:lit-rev}

\begin{figure}[h]
    \centering
    
    \includegraphics[width=0.6\textwidth]{block_diagram.png}
\end{figure}\textbf{}

\begin{equation}
    G(s) = \frac{s+9}{(s^2-6s-16)(s+4)}
\end{equation}

\begin{equation}
    A(s) = \frac{s+4}{s^2+4s+3}
\end{equation}

\begin{equation}
    B(s) = s^2+26s+168
\end{equation}

%===========================================================
\section{Methodology and Results}\label{sec:meth}

\begin{itemize}

\item \textbf{Part 1}\\
\begin{enumerate}

    \item
    ($G(s), A(s),$ and $B(s)$ in factored form, isolating the poles and zeros)
    
    $$G(s) = \frac{s+9}{(s-8)(s+2)(s+4)}$$
    poles: 8, -2, -4\\
    zeros: -9
    
    $$A(s) = \frac{s+4}{(s+1)(s+3)}$$
    poles: -1, -3\\
    zeros: -4
    
    $$B(s) = (s+12)(s+14)$$
    zeros: -12, -14

%===========================================================
    \item 
    Using scipy.signal.tf2zpk() to verify our result for \textbf{part1-(1)}.\\
    
\begin{lstlisting}[language=Python, caption={using scipy.signal.tf2zpk()}, label={lst:code}, mathescape=true, breaklines=true]
# ==================== part 1 - Task 2 ====================
numG, denG = [1, 9], [1, -2, -40, -64]

Gz, Gp, Gk = sig.tf2zpk(numG, denG)                  
print("G(s):", "\nzeros: ", Gz, "\npoles: ", Gp)

numA, denA = [1, 4], [1, 4, 3]

Az, Ap, Ak = sig.tf2zpk(numA, denA)
print("\nA(s):", "\nzeros: ", Az, "\npoles: ", Ap)

numB = [1, 26, 168]

Bz = np.roots(numB)
print("\nB(s):", "\nzeros: ", Bz)
\end{lstlisting}
    
\begin{figure}[h]
    \centering
    \includegraphics[width=0.8\textwidth]{1-task2.jpg}
\end{figure}\textbf{}

As can be seen from the printed output, the poles and zeros we got using the scipy.signal.tf2zpk() function verifies our result from \textbf{part1-(1)}.\\

%===========================================================
    \item
    open-loop transfer function
    
    $$H_{o}(s) = \frac{Y(s)}{X(s)} = A(s)G(s) = \frac{s+9}{(s-8)(s+1)(s+2)(s+3)}$$\\

%===========================================================
    \item 
    From the expression we found for the open-loop in \textbf{part1-(3)}, we can say that the function response will be unstable since it has a positive pole.

\clearpage
    
%===========================================================
    \item
    Using scipy.signal.convolve() to expand the numerator and denominator of the open-loop transfer function then plotting its step response.\\
    
\begin{lstlisting}[language=Python, caption={open-loop step response using scipy.signal.convolve()}, label={lst:code}, mathescape=true, breaklines=true]
# ==================== part 1 - Task 5 ====================
numO = sig.convolve(numG, numA)
denO = sig.convolve(denG, denA)

print("\nOpen-loop transfer function:")
print("Numerator: ", numO, "\nDenominator: ", denO)

steps = 1e-2

t = np.arange(0 , 2+steps , steps)
tout, yout = sig.step((numO, denO), T = t)
\end{lstlisting}

The resulted output and plot:

\begin{figure}[h]
    \centering
    \includegraphics[width=0.6\textwidth]{1-task5.jpg}
\end{figure}\textbf{}
    
\begin{figure}[h]
    \centering
    \includegraphics[width=0.8\textwidth]{figure1.png}
\end{figure}\textbf{}

%===========================================================
    \item
    As can be seen from the open-loop step response plot in \textbf{part1-(5)}, the response grows exponentially and thus supporting what we concluded from the transfer function poles in \textbf{part1-(4)}; that is the open-loop transfer function is unstable. 
    
\end{enumerate}
\clearpage
    
%===========================================================
\item \textbf{Part 2}\\

\begin{enumerate}

    \item
    Closed-loop transfer function symbolically in terms of $numG$ and $denG$,
    
    $$H_{c}(s) = \frac{Y(s)}{X(s)} = \frac{A(s)G(s)}{1+B(s)G(s)} = \frac{numA \cdot numG}{denA \cdot denG + B \cdot denA \cdot numG}$$\\
    
%===========================================================
    \item 
    Using scipy.signal.convolve() and scipy.signal.tf2zpk() function to find the transfer function of the closed-loop.\\

\begin{lstlisting}[language=Python, caption={closed-loop transfer function}, label={lst:code}, mathescape=true, breaklines=true]
# ==================== part 2 - Task 2 ====================
numC = sig.convolve(numA, numG)
denc0 = sig.convolve(denA, denG)
denc1 = sig.convolve(denA, numG)
denc2 = sig.convolve(denc1, numB)

denC = denc0 + denc2

print("\nClosed-loop transfer function:")
print("Numerator: ", numC, "\nDenominator: ", denC)

Cz, Cp, Ck = sig.tf2zpk(numC, denC)
print("\nnzeros: ", Cz, "\npoles: ", Cp)
\end{lstlisting}

The printed output,
\begin{figure}[h]
    \centering
    \includegraphics[width=0.7\textwidth]{2-task2.jpg}
\end{figure}\textbf{}

Thus,
    $$H_{c}(s) = \frac{(s+9)(s+4)}{(s+5.16-9.52j)(s+5.16+9.52j)(s+6.18)(s+3)(s+1)}$$
\newline

%===========================================================
    \item 
    We can see that the response of the closed-loop transfer function will be stable since all of its poles are negative.\\
    
%===========================================================
    \item Step response of the closed-loop transfer function using scipy.signal.step().\\
    
\begin{lstlisting}[language=Python, caption={step response of the closed-loop}, label={lst:code}, mathescape=true, breaklines=true]
# ==================== part 2 - Task 4 ====================

t = np.arange(0 , 10+steps , steps)
tout, yout = sig.step((numC, denC), T = t)
\end{lstlisting}
\clearpage

The resulted plot,
\begin{figure}[h]
    \centering
    \includegraphics[width=0.7\textwidth]{figure2.png}
\end{figure}\textbf{}
%===========================================================
    \item 
    The resulted plot from \textbf{part2-4} shows that the closed-loop response is bounded and thus supporting our concluded answer from the transfer function poles in \textbf{part2-3}; that is the system response is stable.
    
\end{enumerate}
\end{itemize}

%===========================================================

\section{Questions}\label{sec:res}
\begin{enumerate}
    
    \item
    In \textbf{Part 1 Task 5}, why does convolving the factored terms using scipy.signal.convolve() result in the expanded form of the numerator and denominator? Would this work with your user-defined convolution function from \textbf{Lab 3}? Why or why not?\\
    
    $$\mathcal{L}\{f(t)*h(t)\} = F(s)H(s)$$
    
    Thus, convolution in the s-domain is the same as multiplication and the scipy.signal.convolve() identifies this. On the other hand, our user-defined convolution function from \textbf{lab 3} will not work as it was defined to work in the time-domain.\\\\
    Testing scipy.signal.convolve() and our user-defined convolution function for the open-loop transfer function, we get
   
\begin{figure}[h]
    \centering
    \includegraphics[width=0.6\textwidth]{Q1.jpg}
\end{figure}\textbf{}

    \item
    Discuss the difference between the open- and closed-loop systems from \textbf{Part 1} and \textbf{Part 2}. How does stability differ for each case, and why?\\
    
    Considering the open-loop from \textbf{part 1}, $H_{o}$ is unstable because it will have a time domain term such that $e^{t\cdot positive pole} u(t)$. Therefore, the other terms will not matter and the system will be unstable.\\
    
    Considering the closed-loop from \textbf{part 2}, $H_{c}$ is stable because all of its poles are in the left half of the plane.\\
    
    \item
    What is the difference between scipy.signal.residue() used in \textbf{Lab 6} and
    scipy.signal.tf2zpk() used in this lab?\\
    
    scipy.signal.residue() is used to compute partial fraction expansion.\\
    
    scipy.signal.tf2zpk() is used for determining the zeros, poles, and gain of a transfer function.\\
    
    \item
    Is it possible for an open-loop system to be stable? What about for a closed-loop system to be unstable? Explain how or how not for each.\\
    
    Both open-loop and closed-loop systems can be either stable or unstable; what determines if a system is stable or unstable is whether the system will decay (stable) or keep growing without an end (unstable).\\
    
    \item
    Leave any feedback on the clarity/usefulness of the purpose, deliverables, and expectations for this lab.\\
    The lab instructions were clear and precise. 

\end{enumerate}
%===========================================================   
\section{Conclusion}\label{sec:res}
    
    By completing the lab we have learned how to judge a system stability by looking at its factored form. We have also familiarized ourselves with Laplace-domain block diagrams and the use of the scipy.signal.tf2zpk() function.
 
%\lipsum[7-8]\cite{knuthwebsite}
%===========================================================
%===========================================================
\bibliographystyle{ieeetr}
\bibliography{refs}
\end{document} 