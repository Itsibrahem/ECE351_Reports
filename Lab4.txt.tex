\documentclass[12pt,a4paper]{article}
\usepackage[utf8]{inputenc}
\usepackage[greek,english]{babel}
\usepackage{alphabeta} 
\usepackage[pdftex]{graphicx}
\usepackage[top=1in, bottom=1in, left=1in, right=1in]{geometry}
\linespread{1.06}
\setlength{\parskip}{8pt plus2pt minus2pt}
\widowpenalty 10000
\clubpenalty 10000
\newcommand{\eat}[1]{}
\newcommand{\HRule}{\rule{\linewidth}{0.5mm}}
\usepackage[official]{eurosym}
\usepackage{enumitem}
\setlist{nolistsep,noitemsep}
\usepackage[hidelinks]{hyperref}
\usepackage{cite}
\usepackage{lipsum}

%===========================================================
\usepackage{graphicx}
\usepackage{listings}

\renewcommand\lstlistingname{Source Code}

\usepackage{xcolor}
\definecolor{codegreen}{rgb}{0,0.6,0}
\definecolor{codegray}{rgb}{0.5,0.5,0.5}
\definecolor{codeblue}{rgb}{0,0,0.95}
\definecolor{backcolour}{rgb}{0.95,0.95,0.92}

\lstdefinestyle{mystyle}{
   backgroundcolor=\color{backcolour},   
   commentstyle=\color{codegreen},
   keywordstyle=\color{codeblue},
   numberstyle=\tiny\color{codegray},
   stringstyle=\color{codegreen},
   basicstyle=\ttfamily\footnotesize,
   breakatwhitespace=false,         
   breaklines=true,                 
   captionpos=b,                    
   keepspaces=true,                 
   numbers=left,                    
   numbersep=5pt,                  
   showspaces=false,                
   showstringspaces=false,
   showtabs=false,                  
   tabsize=2
   xleftmargin=10pt
}
\lstset{style=mystyle}

%===========================================================
\begin{document}
%===========================================================
\begin{titlepage}
\begin{center}
% Top 
\includegraphics[width=0.55\textwidth]{cut-logo-en}~\\[2cm]
% Title
\HRule \\[0.4cm]
{ \LARGE 
  \textbf{Lab 4 Project Report for ECE 351}\\[0.4cm]
  \emph{System Step Response Using Convolution}\\[0.4cm]
}
\HRule \\[1.5cm]
% Author
{ \large
  Ibrahem Alobaid \\[0.1cm]
  September 22, 2022\\[0.1cm]
  %#\texttt{user@cut.ac.cy}
}
\vfill
%\textsc{\Large Cyprus University of Technology}\\[0.4cm]\textsc{\large Department of Electrical Engineering,\\Computer Engineering \& Informatics}\\[0.4cm]
% Bottom

\end{center}
\end{titlepage}
%\begin{abstract}
%\lipsum[1-2]
%\addtocontents{toc}{\protect\thispagestyle{empty}}
%\end{abstract}
\newpage
%===========================================================
\tableofcontents
\addtocontents{toc}{\protect\thispagestyle{empty}}
\newpage
\setcounter{page}{1}
%===========================================================
%===========================================================
\section{Introduction}\label{sec:intro}
    This lab purpose was to make us familiar with how to use convolution to compute a system's step response.
%===========================================================

\section{Equations}\label{sec:lit-rev}
    The following signals $(h_1(t), h_2(t), h_3(t))$ were defined and plotted. We have also used the defined signals to convolve them with a step function (using the convolution function we defined in \textbf{Lab 3} and by solving the convolution integral).

\begin{equation}
    h_1(t) = e^{-2t}[u(t) - u(t-3)]
\end{equation}
\begin{equation}
    h_2(t) = u(t-2) - u(t-6)
\end{equation}
\begin{equation}
    h_3(t) = 
\end{equation}

    \\ Hand-calculated step response of the three transfer function:
    
\begin{equation}
    h_1(t)*u(t) = \frac{1}{2}(1-e^{-2t})u(t) - \frac{1}{2}(1-e^{-2(t-3)})u(t-3)
\end{equation}
\begin{equation}
    h_2(t)*u(t) = (t-2)u(t-2)-(t-6)u(t-6)
\end{equation}
\begin{equation}
    h_3(t)*u(t) = \frac{1}{w}sin(wt)u(t)
\end{equation}  
%===========================================================
\section{Methodology and Results}\label{sec:meth}
\begin{itemize}
    \item
    Part 1:\\
    
    \\For the first part, we created three signals (as shown in equation 1, 2, and 3) as user-defined functions and plotted them.
    
\begin{lstlisting}[language=Python, caption={User-defined functions}, label={lst:code}, mathescape=true, breaklines=true]
t = np.arange(-10, 10 + steps , steps)

f = 0.25
w = 2 * math.pi * f

def h1(t):
    return (np.exp(-2*t)) * (u(t) - u(t-3))

def h2(t):
    return u(t-2) - u(t-6)

def h3(t):
    return np.cos(w*t) * u(t)
    
plt.figure(figsize = (10, 7))
plt.subplot(3 , 1 , 1)
plt.plot(t , h1(t))
plt.grid()
plt.ylabel('h1(t)')
plt.title('Figure 1')
 
plt.subplot(3 , 1 , 2)
plt.plot(t , h2(t))
plt.grid()
plt.ylabel('h2(t)')
 
plt.subplot(3 , 1 , 3)
plt.plot(t , h3(t))
plt.grid()
plt.ylabel('h3(t)')
plt.xlabel('t')
\end{lstlisting}

    \\The resulting plots of $(h_1(t), h_2(t), h_3(t))$:
\begin{figure}[h]
    \centering
    \includegraphics[width=0.7\textwidth]{part1.png}
\end{figure}

%===========================================================
    \item
    Part 2:\\
    
    \\For this part, we used the convolution function we defined in \textbf{Lab 3} to plot the step response of $(h_1(t), h_2(t), h_3(t))$. We have also plotted the step response of the three functions by solving the convolution integral (as shown in equations 4, 5, and 6).

\begin{lstlisting}[language=Python, caption={Step response using user-defined function}, label={lst:code}, mathescape=true, breaklines=true]
def my_conv(h1, h2):
    
    Nh1 = len(h1)
    Nh2 = len(h2)
     
    h1Extended = np.append(h1, np.zeros((1, Nh2-1))) 
    h2Extended = np.append(h2, np.zeros((1, Nh1-1)))
     
    result = np.zeros(h1Extended.shape)
     
    for i in range((Nh2 + Nh1)-2): 
        result[i] = 0  
          
        for j in range(Nh1): 
            if ((i-j)+1 > 0): 
                try: 
                    result[i] += h1Extended[j] * h2Extended[i-j+1]
                except:
                    print(i-j)                 
    return result

y1 = my_conv(h1(t), u(t)) 
y2 = my_conv(h2(t), u(t)) 
y3 = my_conv(h3(t), u(t))

tExtended = np.arange(2*t[0], 2*t[len(t)-1]+steps, steps)

plt.figure(figsize = (10, 7))
plt.subplot(3 , 1 , 1)
plt.plot(tExtended, y1)
plt.grid()
plt.ylabel('Responce1')
plt.title('Figure 2')

plt.subplot(3 , 1 , 2)
plt.plot(tExtended, y2)
plt.grid()
plt.ylabel('Responce2')

plt.subplot(3 , 1 , 3)
plt.plot(tExtended, y3)
plt.grid()
plt.ylabel('Responce3')
plt.xlabel('t')
\end{lstlisting}
\clearpage

    The resulting plots of the user-defined convolution function:\\
\begin{figure}[h]
    \centering
    \includegraphics[width=0.8\textwidth]{part2(a).png}
\end{figure}\textbf{}

\begin{lstlisting}[language=Python, caption={Step response by hand}, label={lst:code}, mathescape=true, breaklines=true]
y1c =  0.5*(1-np.exp(-2*tExtended))*u(tExtended)-0.5*(1-np.exp(-2*(tExtended-3)))*u(tExtended-3)
y2c = ((tExtended-2) * u(tExtended-2)) - ((tExtended-6) * u(tExtended-6))
y3c = (1 / w) * (np.sin(w*tExtended) * u(tExtended))

plt.figure( figsize = (10 , 7))
plt.subplot(3, 1, 1)
plt.plot(tExtended, y1c)
plt.grid()
plt.ylabel('Responce1')
plt.title('Step Respnoce (Hand Calculated)')

plt.subplot(3, 1, 2)
plt.plot(tExtended, y2c)
plt.grid()
plt.ylabel('Responce2')

plt.subplot(3, 1, 3)
plt.plot(tExtended, y3c)
plt.grid()
plt.ylabel('Responce3')
plt.xlabel('t')

plt.show()
\end{lstlisting}
\clearpage

    The resulting plots of the hand calculated convolution:\\
\begin{figure}[h]
    \centering
    \includegraphics[width=0.7\textwidth]{part2(b).png}
\end{figure}\textbf{}

\end{itemize}
%===========================================================

\section{Questions}\label{sec:res}
\begin{enumerate}

    \item
    Leave any feedback on the clarity of lab tasks, expectations, and deliverables.\\
    \\The lab procedure and purpose were very clear.
\end{enumerate}
%===========================================================   
\section{Conclusion}\label{sec:res}
    In this lab, we have familiarized ourselves with using convolution to compute the step response of a transfer function.
%\lipsum[7-8]\cite{knuthwebsite}
%===========================================================
%===========================================================
\bibliographystyle{ieeetr}
\bibliography{refs}
\end{document} 
