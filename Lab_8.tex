\documentclass[12pt,a4paper]{article}
\usepackage[utf8]{inputenc}
\usepackage[greek,english]{babel}
\usepackage{alphabeta} 
\usepackage[pdftex]{graphicx}
\usepackage[top=1in, bottom=1in, left=1in, right=1in]{geometry}
\linespread{1.06}
\setlength{\parskip}{8pt plus2pt minus2pt}
\widowpenalty 10000
\clubpenalty 10000
\newcommand{\eat}[1]{}
\newcommand{\HRule}{\rule{\linewidth}{0.5mm}}
\usepackage[official]{eurosym}
\usepackage{enumitem}
\setlist{nolistsep,noitemsep}
\usepackage[hidelinks]{hyperref}
\usepackage{cite}
\usepackage{lipsum}

%===========================================================
\usepackage{graphicx}
\usepackage{listings}

\renewcommand\lstlistingname{Source Code}

\usepackage{xcolor}
\definecolor{codegreen}{rgb}{0,0.6,0}
\definecolor{codegray}{rgb}{0.5,0.5,0.5}
\definecolor{codeblue}{rgb}{0,0,0.95}
\definecolor{backcolour}{rgb}{0.95,0.95,0.92}

\lstdefinestyle{mystyle}{
   backgroundcolor=\color{backcolour},   
   commentstyle=\color{codegreen},
   keywordstyle=\color{codeblue},
   numberstyle=\tiny\color{codegray},
   stringstyle=\color{codegreen},
   basicstyle=\ttfamily\footnotesize,
   breakatwhitespace=false,         
   breaklines=true,                 
   captionpos=b,                    
   keepspaces=true,                 
   numbers=left,                    
   numbersep=5pt,                  
   showspaces=false,                
   showstringspaces=false,
   showtabs=false,                  
   tabsize=2
   xleftmargin=10pt
}
\lstset{style=mystyle}

%===========================================================
\begin{document}
%===========================================================
\begin{titlepage}
\begin{center}
% Top 
\includegraphics[width=0.55\textwidth]{cut-logo-en}~\\[2cm]
% Title
\HRule \\[0.4cm]
{ \LARGE 
  \textbf{Lab 8 Project Report for ECE 351}\\[0.4cm]
  \emph{Fourier Series Approximation of a Square Wave}\\[0.4cm]
}
\HRule \\[1.5cm]
% Author
{ \large
  Ibrahem Alobaid \\[0.1cm]
  October 20, 2022\\[0.1cm]
  %#\texttt{user@cut.ac.cy}
}
\vfill
%\textsc{\Large Cyprus University of Technology}\\[0.4cm]\textsc{\large Department of Electrical Engineering,\\Computer Engineering \& Informatics}\\[0.4cm]
% Bottom

\end{center}
\end{titlepage}
%\begin{abstract}
%\lipsum[1-2]
%\addtocontents{toc}{\protect\thispagestyle{empty}}
%\end{abstract}
\newpage
%===========================================================
\tableofcontents
\addtocontents{toc}{\protect\thispagestyle{empty}}
\newpage
\setcounter{page}{1}
%===========================================================
%===========================================================
\section{Introduction}\label{sec:intro}
    In this lab, our purpose was to use Fourier series to approximate periodic time-domain signals.
%===========================================================

\section{Equations}\label{sec:lit-rev}

In this lab, we were given the following square wave function,

\begin{figure}[h]
    \centering
    \includegraphics[width=0.8\textwidth]{squareWave.jpg}
    \\ Figure 1: square wave function
\end{figure}\textbf{}

that can be represented by the Fourier series,

    $$x(t) = \sum_{k=1}^{\infty} \frac{2}{k\pi} [1-cos(k\pi)]sin(k\omega_{0}t)$$

where,
    
    $$\omega_{0} = \frac{2\pi}{T}$$
\clearpage
%===========================================================
\section{Methodology and Results}\label{sec:meth}

\begin{itemize}

\item \textbf{Part 1}\\
\begin{enumerate}

    \item
    For the first part, we were asked to input the expression for $a_{k}$ and $b_{k}$ into Python and print out the numerical values of $a_{0}, a_{1}, b_{1}, b_{2},$ and $ b_{3}.$\\

\begin{lstlisting}[language=Python, caption={Task 1, Part 1}, label={lst:code}, mathescape=true, breaklines=true]
# ==================== Part 1 - Task 1 ====================
a = np.zeros((4, 1))
b = np.zeros((4, 1))

for k in np.arange(1, 4):
    b[k] = 2/(k*np.pi) * (1-np.cos(k*np.pi))

for k in np.arange(1, 4):
    a[k] = 0
    
print("a_0 = ", a[0], "\na_1 = ", a[1])
print("b_1 = ", b[1], "\nb_2 = ", b[2], "\nb_3 = ", b[3])
\end{lstlisting}

The printed values,

\begin{figure}[h]
    \centering
    \includegraphics[width=0.3\textwidth]{task11.jpg}
\end{figure}\textbf{}

%===========================================================
    \item
    Our second task was to plot the square wave Fourier series approximation for $N = \{1, 3, 15, 50, 150,1500\}$; with $T = 8s$ and plot approximation time for $0 \leq t \leq 20s$ using appropriate step size.\\

\begin{lstlisting}[language=Python, caption={Task 2, Part 1}, label={lst:code}, mathescape=true, breaklines=true]
# ==================== Part 1 - Task 2 ====================
T = 8
w = 2*np.pi/T
steps = 1e-3
t=np.arange(0 , 20 + steps , steps)

N = [1, 3, 15, 50, 150, 1500]
y = [0, 0, 0, 0, 0, 0]

for i in range(len(N)):
    for k in range(N[i]):
        b = 2/((k+1)*np.pi) * (1-np.cos((k+1)*np.pi))
        x = b * np.sin((k+1)*w*t)
        
        y[i] += x
\end{lstlisting}
\clearpage

The resulting plots,

\begin{figure}[h]
    \centering
    \includegraphics[width=0.7\textwidth]{fig1.png}
\end{figure}\textbf{}

\begin{figure}[h]
    \centering
    \includegraphics[width=0.7\textwidth]{fig2.png}
\end{figure}\textbf{}

\end{enumerate}
\end{itemize}
\clearpage

%===========================================================
\section{Questions}\label{sec:res}
\begin{enumerate}
    \item Is x(t) an even or an odd function? Explain why.\\
    \\$x(t)$ is odd as $x_{-n} = -x_{n}$\\
    
    \item Based on your results from Task 1, what do you expect the values of a2, a3, . . . , an to be? Why?\\
    \\The values of $a_{2}, a_{3},..., a_{n}$ will all be zeroes as $x(t)$ is an odd function.\\ 
    
    \item How does the approximation of the square wave change as the value of N increases? In what way does the Fourier series struggle to approximate the square wave?\\
    \\As the value of N increases, we get better approximation of the square wave.
    \\The Fourier series struggle in that it need to sums infinite number of $N$ terms to be exact to the square wave.\\  
    
    \item What is occurring mathematically in the Fourier series summation as the value of N increases?\\
    \\As the value of $N$ increases, the more sinusoidal wave we are adding together; thus, resulting in better approximation of the square wave.\\
    
    \item Leave any feedback on the clarity of lab tasks, expectations, and deliverables.\\
    \\The lab instructions were clear and precise.
\end{enumerate}

%===========================================================   
\section{Conclusion}\label{sec:res}
    
    By completing the lab we have learned how to use Fourier series to approximate periodic time-domain signals.
 
%\lipsum[7-8]\cite{knuthwebsite}
%===========================================================
%===========================================================
\bibliographystyle{ieeetr}
\bibliography{refs}
\end{document} 