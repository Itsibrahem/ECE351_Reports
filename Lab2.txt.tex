\documentclass[12pt,a4paper]{article}
\usepackage[utf8]{inputenc}
\usepackage[greek,english]{babel}
\usepackage{alphabeta} 
\usepackage[pdftex]{graphicx}
\usepackage[top=1in, bottom=1in, left=1in, right=1in]{geometry}
\linespread{1.06}
\setlength{\parskip}{8pt plus2pt minus2pt}
\widowpenalty 10000
\clubpenalty 10000
\newcommand{\eat}[1]{}
\newcommand{\HRule}{\rule{\linewidth}{0.5mm}}
\usepackage[official]{eurosym}
\usepackage{enumitem}
\setlist{nolistsep,noitemsep}
\usepackage[hidelinks]{hyperref}
\usepackage{cite}
\usepackage{lipsum}

%===========================================================
\usepackage{graphicx}
\usepackage{listings}

\renewcommand\lstlistingname{Source Code}

\usepackage{xcolor}
\definecolor{codegreen}{rgb}{0,0.6,0}
\definecolor{codegray}{rgb}{0.5,0.5,0.5}
\definecolor{codeblue}{rgb}{0,0,0.95}
\definecolor{backcolour}{rgb}{0.95,0.95,0.92}

\lstdefinestyle{mystyle}{
   backgroundcolor=\color{backcolour},   
   commentstyle=\color{codegreen},
   keywordstyle=\color{codeblue},
   numberstyle=\tiny\color{codegray},
   stringstyle=\color{codegreen},
   basicstyle=\ttfamily\footnotesize,
   breakatwhitespace=false,         
   breaklines=true,                 
   captionpos=b,                    
   keepspaces=true,                 
   numbers=left,                    
   numbersep=5pt,                  
   showspaces=false,                
   showstringspaces=false,
   showtabs=false,                  
   tabsize=2
   xleftmargin=10pt
}
\lstset{style=mystyle}

%===========================================================
\begin{document}
%===========================================================
\begin{titlepage}
\begin{center}
% Top 
\includegraphics[width=0.55\textwidth]{cut-logo-en}~\\[2cm]
% Title
\HRule \\[0.4cm]
{ \LARGE 
  \textbf{Lab 2 Project Report for ECE 351}\\[0.4cm]
  \emph{User-Defined Functions}\\[0.4cm]
}
\HRule \\[1.5cm]
% Author
{ \large
  Ibrahem Alobaid \\[0.1cm]
  September 8, 2022\\[0.1cm]
  %#\texttt{user@cut.ac.cy}
}
\vfill
%\textsc{\Large Cyprus University of Technology}\\[0.4cm]\textsc{\large Department of Electrical Engineering,\\Computer Engineering \& Informatics}\\[0.4cm]
% Bottom

\end{center}
\end{titlepage}
%\begin{abstract}
%\lipsum[1-2]
%\addtocontents{toc}{\protect\thispagestyle{empty}}
%\end{abstract}
\newpage
%===========================================================
\tableofcontents
\addtocontents{toc}{\protect\thispagestyle{empty}}
\newpage
\setcounter{page}{1}
%===========================================================
%===========================================================
\section{Introduction}\label{sec:intro}
This lab purpose was to introduce us to user-defined functions in Python. In addition, we will familiarize ourselves with different types of signal operations. Those operations includes time shifting, time scaling, time reversal, signal addition, and discrete differentiation.
%===========================================================

\section{Equations}\label{sec:lit-rev}
To achieve the graph shown in the lab manual (titled "Plot for Lab2"), we had to derive the following function:
\begin{equation}
    y = r(t) - r(t-3) + 5*step(t-3) - 2*step(t-6) - 2*r(t-6)
\end{equation}

%===========================================================
\section{Methodology and Results}\label{sec:meth}
\begin{itemize}
    \item
    Part 1:
    
    For the first part, we only needed to do miner changes to the given example code to achieve our user-defined function which is a cosine plot. 
    
\begin{lstlisting}[language=Python, caption={User-defined function}, label={lst:code}, mathescape=true, breaklines=true]
import numpy as np
import matplotlib.pyplot as plt

plt.rcParams['font.size'] = 14 

steps = 1e-2
t = np.arange(0, 10 + steps , steps)

def func1(t): 
    y = np.zeros(t.shape) 
    
    for i in range(len(t)): # loop from 0 to the size of the array
            y[i] = np.cos(t[i])
    return y
    
y = func1(t) # invoke func1

plt.figure(figsize = (10, 7))
plt.plot(t, y)
plt.grid()

plt.ylabel('y(t)')
plt.xlabel('t')
plt.title('cosine function')

plt.show()
\end{lstlisting}
\clearpage

    The resulting graph of the defined function:\\
\begin{figure}[h]
    \centering
    \includegraphics[width=0.8\textwidth]{Part1_plot.png}
\end{figure}
%===========================================================
    \item
    Part 2:
    
    For this part, we needed to derive an equation to achieve the desired plot. Our equation used both a step and a ramp function. Thus, those two function were defined first to accomplish a plot of the derived equation.

\begin{lstlisting}[language=Python, caption={Step function}, label={lst:code}, mathescape=true, breaklines=true]
def step(t):
    y = np.zeros(t.shape)
    
    for i in range(len(t)):
        if t[i] <= 0:
            y[i] = 0
        else:
            y[i] = 1
    return y
    
y = step(t)
\end{lstlisting}
\clearpage

    The resulting plot of the defined function:\\
\begin{figure}[h]
    \centering
    \includegraphics[width=0.8\textwidth]{step_plot.png}
\end{figure}\textbf{}

\begin{lstlisting}[language=Python, caption={Ramp function}, label={lst:code}, mathescape=true, breaklines=true]
def r(t):
    y = np.zeros(t.shape)
    
    for i in range(len(t)):
        if t[i] <= 0:
            y[i] = 0
        else:
            y[i] = t[i]
    return y
    
y = r(t)
\end{lstlisting}
\clearpage

    The resulting plot of the defined function:\\
\begin{figure}[h]
    \centering
    \includegraphics[width=0.7\textwidth]{ramp_plot.png}
\end{figure}\textbf{}
    
    Using both defined functions (step and ramp) we can model a simple plot by:
\begin{lstlisting}[language=Python, caption={Utilizing the user-defined functions}, label={lst:code}, mathescape=true, breaklines=true]
y = r(t) - r(t-3) + 5*step(t-3) - 2*step(t-6) - 2 * r(t-6)
\end{lstlisting}

    and thus, we achieve the wanted plot:\\
\begin{figure}[h]
    \centering
    \includegraphics[width=0.8\textwidth]{step_ramp_plot.png}
\end{figure}\textbf{}
\clearpage
%===========================================================
    \item
    Part 3:
    
    For this part, we tested our function from part by applying signal operations to it.
\begin{lstlisting}[language=Python, caption={Time reversal}, label={lst:code}, mathescape=true, breaklines=true]
def func(t):
    return r(t) - r(t-3) + 5*step(t-3) - 2*step(t-6) - 2 * r(t-6)
    
y = func(-t)
\end{lstlisting}

\begin{figure}[h]
    \centering
    \includegraphics[width=0.6\textwidth]{time_reversal.png}
\end{figure}\textbf{}

\begin{lstlisting}[language=Python, caption={Time Shift f(t-4)}, label={lst:code}, mathescape=true, breaklines=true]
def func(t):
    return r(t) - r(t-3) + 5*step(t-3) - 2*step(t-6) - 2 * r(t-6)
    
y = func(t-4)
\end{lstlisting}

\begin{figure}[h]
    \centering
    \includegraphics[width=0.6\textwidth]{time_shift_1.png}
\end{figure}\textbf{}

\begin{lstlisting}[language=Python, caption={Time Shift f(-t-4)}, label={lst:code}, mathescape=true, breaklines=true]
def func(t):
    return r(t) - r(t-3) + 5*step(t-3) - 2*step(t-6) - 2 * r(t-6)
    
y = func(-t-4)
\end{lstlisting}

\begin{figure}[h]
    \centering
    \includegraphics[width=0.6\textwidth]{time_shift_2.png}
\end{figure}\textbf{}

\begin{lstlisting}[language=Python, caption={Time scale f(t/2)}, label={lst:code}, mathescape=true, breaklines=true]
def func(t):
    return r(t) - r(t-3) + 5*step(t-3) - 2*step(t-6) - 2 * r(t-6)
    
y = func(t/2)
\end{lstlisting}

\begin{figure}[h]
    \centering
    \includegraphics[width=0.6\textwidth]{time_scale1.png}
\end{figure}\textbf{}
\clearpage

\begin{lstlisting}[language=Python, caption={Time scale f(2t)}, label={lst:code}, mathescape=true, breaklines=true]
def func(t):
    return r(t) - r(t-3) + 5*step(t-3) - 2*step(t-6) - 2 * r(t-6)
    
y = func(2*t)
\end{lstlisting}

\begin{figure}[h]
    \centering
    \includegraphics[width=0.6\textwidth]{time_scale2.png}
\end{figure}\textbf{}

    Hand-drawn derivative plot:\\
\begin{figure}[h]
    \centering
    \includegraphics[width=0.6\textwidth]{Hand-drawn derivative plot.png}
\end{figure}\textbf{}
\clearpage

\begin{lstlisting}[language=Python, caption={Derivative of user-defined function}, label={lst:code}, mathescape=true, breaklines=true]
dy = np.diff(func(t))

\end{lstlisting}

\begin{figure}[h]
    \centering
    \includegraphics[width=0.6\textwidth]{derivative_plot.png}
\end{figure}\textbf{}

\end{itemize}
%===========================================================
\section{Error Analysis}\label{sec:res}
    For this lab, I found user-defined functions to be somewhat straight forward. I mostly struggled with part 3 - task 5, that is taking the derivative of the function we defined using the numpy.diff() function.
%===========================================================
\section{Questions}\label{sec:res}
\begin{enumerate}
    \item
    Are the plots from Part 3 Task 4 and Part 3 Task 5 identical? Is it possible for them to
    match? Explain why or why not.\\
    No, the plots are not identical and that is because we hand drew the plot of the function by taking the slope. On the other hand, numpy.diff() works by taking the difference between two object in the array. 
    
    \item
    How does the correlation between the two plots (from Part 3 Task 4 and Part 3 Task 5)
    change if you were to change the step size within the time variable in Task 5? Explain why
    this happens.\\
    (For me) changing the step size within the time variable and running the program resulted in no plot. This might be because with smaller step size, it will be like taking the derivative of infinite number of points. 
    
    \item
    Leave any feedback on the clarity of lab tasks, expectations, and deliverables.\\
    Overall, the lab manual was great. I liked that there was an example code to help us with our user-defined functions.
\end{enumerate}
%===========================================================   
\section{Conclusion}\label{sec:res}
    In this lab, we have familiarized ourselves with how to define functions in Python, and utilize them or do signal operations on them. We also learned how to create good resolution plots, and properly display them to the interpreter.
%\lipsum[7-8]\cite{knuthwebsite}
%===========================================================
%===========================================================
\bibliographystyle{ieeetr}
\bibliography{refs}
\end{document} 