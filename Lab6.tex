\documentclass[12pt,a4paper]{article}
\usepackage[utf8]{inputenc}
\usepackage[greek,english]{babel}
\usepackage{alphabeta} 
\usepackage[pdftex]{graphicx}
\usepackage[top=1in, bottom=1in, left=1in, right=1in]{geometry}
\linespread{1.06}
\setlength{\parskip}{8pt plus2pt minus2pt}
\widowpenalty 10000
\clubpenalty 10000
\newcommand{\eat}[1]{}
\newcommand{\HRule}{\rule{\linewidth}{0.5mm}}
\usepackage[official]{eurosym}
\usepackage{enumitem}
\setlist{nolistsep,noitemsep}
\usepackage[hidelinks]{hyperref}
\usepackage{cite}
\usepackage{lipsum}

%===========================================================
\usepackage{graphicx}
\usepackage{listings}

\renewcommand\lstlistingname{Source Code}

\usepackage{xcolor}
\definecolor{codegreen}{rgb}{0,0.6,0}
\definecolor{codegray}{rgb}{0.5,0.5,0.5}
\definecolor{codeblue}{rgb}{0,0,0.95}
\definecolor{backcolour}{rgb}{0.95,0.95,0.92}

\lstdefinestyle{mystyle}{
   backgroundcolor=\color{backcolour},   
   commentstyle=\color{codegreen},
   keywordstyle=\color{codeblue},
   numberstyle=\tiny\color{codegray},
   stringstyle=\color{codegreen},
   basicstyle=\ttfamily\footnotesize,
   breakatwhitespace=false,         
   breaklines=true,                 
   captionpos=b,                    
   keepspaces=true,                 
   numbers=left,                    
   numbersep=5pt,                  
   showspaces=false,                
   showstringspaces=false,
   showtabs=false,                  
   tabsize=2
   xleftmargin=10pt
}
\lstset{style=mystyle}

%===========================================================
\begin{document}
%===========================================================
\begin{titlepage}
\begin{center}
% Top 
\includegraphics[width=0.55\textwidth]{cut-logo-en}~\\[2cm]
% Title
\HRule \\[0.4cm]
{ \LARGE 
  \textbf{Lab 6 Project Report for ECE 351}\\[0.4cm]
  \emph{Partial Fraction Expansion}\\[0.4cm]
}
\HRule \\[1.5cm]
% Author
{ \large
  Ibrahem Alobaid \\[0.1cm]
  October 6, 2022\\[0.1cm]
  %#\texttt{user@cut.ac.cy}
}
\vfill
%\textsc{\Large Cyprus University of Technology}\\[0.4cm]\textsc{\large Department of Electrical Engineering,\\Computer Engineering \& Informatics}\\[0.4cm]
% Bottom

\end{center}
\end{titlepage}
%\begin{abstract}
%\lipsum[1-2]
%\addtocontents{toc}{\protect\thispagestyle{empty}}
%\end{abstract}
\newpage
%===========================================================
\tableofcontents
\addtocontents{toc}{\protect\thispagestyle{empty}}
\newpage
\setcounter{page}{1}
%===========================================================
%===========================================================
\section{Introduction}\label{sec:intro}
    In this lab, our purpose was learn how to perform partial fraction expansion using the scipy.signal.residue() function.
%===========================================================

\section{Equations}\label{sec:lit-rev}

\begin{itemize}
    \item 
    From the pre-lab:
\begin{equation}
    Y(s) = X(s)\frac{s^2+6s+12}{s^2+10s+24} 
\end{equation}

\newline
    $$\rightarrow Y(s) = \frac{1}{s+6}-\frac{0.5}{s+4}+\frac{0.5}{s}$$

\newline
    $$\rightarrow y(t) = [e^{-6t}-0.5e^{-4t}+0.5]u(t)$$

    \item
    From the lab manual:
\begin{equation}
    y^{5}(t)+18y^{(4)}(t)+218y^{(3)}(t)+2036y^{(2)}(t)+9085y^{(1)}(t)+25250y(t) = 25250x(t)
\end{equation}
\end{itemize}

%===========================================================
\section{Methodology and Results}\label{sec:meth}

\begin{itemize}

    \item Part 1:
    Plot of the transfer function y(t) step response that we found in the pre-lab (shown in the equation section). In addition, we plotted the step response of y(t) using the scipy.signal.step() function. The result of the user-defined and scipy.signal.step() plot should be identical verifying that the resulted y(t) from the pre-lab is correct.\\
    
\begin{lstlisting}[language=Python, caption={user-defined and scipy.signal.step() step response}, label={lst:code}, mathescape=true, breaklines=true]
# ==================== part 1 - Task 1 ====================
def y(t):
    y = (np.exp(-6*t) - 0.5*np.exp(-4*t) + 0.5)*u(t)
    return y
    
# ==================== part 1 - Task 2 ====================
numH = [1, 6, 12]
denH = [1, 10, 24]

tout, yout = sig.step((numH, denH), T = t)
\end{lstlisting}
\clearpage

    The resulting plots of the user-defined and scipy.signal.step():
    
\begin{figure}[h]
    \centering
    \includegraphics[width=0.8\textwidth]{figure1.png}
\end{figure}\textbf{}
    
\begin{lstlisting}[language=Python, caption={using scipy.signal.residue() to pereform partial fraction expansion}, label={lst:code}, mathescape=true, breaklines=true]
# ==================== part 1 - Task 3 ====================
denH.append(0)
R, P, K = sig.residue(numH, denH)

print('Part1:', '\nresidue: ', R, '\npoles: ', P, '\ngain: ', K)
\end{lstlisting}
    
    The printed output of the scipy.signal.residue() call:
\begin{figure}[h]
    \centering
    \includegraphics[width=0.3\textwidth]{part1Task3.jpg}
\end{figure}\textbf{}
    
    As can be seen, the printed partial fraction expansion using scipy.signal.residue() agrees with our result from the pre-lab (shown in the equation section).
\clearpage

    \item Part 2:
    For the second part of the lab, we were given a system (shown in the equation section) and we were asked to find the partial fraction expansion of the step response using scipy.signal.residue(). We then used the result we found using the scipy.signal.residue() in a function we defined to perform the cosine method. We also plotted the system time-domain response using our user-defined function and the scipy.signal.step() function. The resulted plots of the defined cosine method and the scipy.signal.step() function should be identical verifying that we properly defined the cosine method. 
    
\begin{lstlisting}[language=Python, caption={partial fraction expansion using scipy.signal.residue()}, label={lst:code}, mathescape=true, breaklines=true]
# ==================== part 2 - Task 1 ====================
numY = [25250]
denY = [1, 18, 218, 2036, 9085, 25250, 0]

R, P, K = sig.residue(numY, denY)

print('\nPart2:', '\nresidue: ', R, '\npoles: ', P, '\ngain: ', K)
\end{lstlisting}

    The printed output of scipy.signal.residue(): 
\begin{figure}[h]
    \centering
    \includegraphics[width=0.9\textwidth]{part2Task1.jpg}
\end{figure}\textbf{}

\begin{lstlisting}[language=Python, caption={user-defined and scipy.signal.step() step response}, label={lst:code}, mathescape=true, breaklines=true]
# ==================== part 2 - Task 2 ====================
t = np.arange(0 , 4.5+steps , steps)

def cosine(R, P, K):
    y = 0
    
    for i in range(len(R)):
        R_mag = np.abs(R[i])
        R_ang = np.angle(R[i])
        alpha = np.real(P[i])
        omega = np.imag(P[i])
        
        y+=(R_mag*np.exp(alpha*t) * np.cos(omega*t + R_ang) * u(t))
        
    return y

# ==================== part 2 - Task 3 ====================
denY.pop()

tout, yout = sig.step((numY, denY), T = t)
\end{lstlisting}
\clearpage

    The resulting plots of the user-defined and scipy.signal.step():
\begin{figure}[h]
    \centering
    \includegraphics[width=0.9\textwidth]{figure2.png}
\end{figure}\textbf{}
\end{itemize}


%===========================================================

\section{Questions}\label{sec:res}
\begin{enumerate}
    
    \item
    For a non-complex pole-residue term, you can still use the cosine method, explain why this
    works.\\
    $$y_{c}(t) = |k|e^{\alpha t}cos(\omega t + \angle k)u(t)$$
    
    \\Rewriting the formal in term of P, R\\
    
    $$y_{c}(t) = |R|e^{P_{real-part}t}cos(P_{imaginary-part}t + \angle R)u(t)$$
    
    \\As can be seen from above the non-complex pole-residue term only effects the frequency $(\omega)$, and thus if zero will result in a constant.\\
    
    \item
    Leave any feedback on the clarity of the expectations, instructions, and deliverables.\\
    \\The lab purpose, deliverable, and tasks were clear and precise.
\end{enumerate}
%===========================================================   
\section{Conclusion}\label{sec:res}
 
    By performing this lab, we have learned how to perform partial fraction expansion in Python by using the scipy.signal.residue() function. In addition, we have learned how to verify our derived result using some of the functions from the scipy.signal library. 
%\lipsum[7-8]\cite{knuthwebsite}
%===========================================================
%===========================================================
\bibliographystyle{ieeetr}
\bibliography{refs}
\end{document} 